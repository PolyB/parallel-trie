\documentclass{article}
\usepackage[utf8]{inputenc}
\usepackage[T1]{fontenc}
\usepackage[francais]{babel}

\usepackage[nameinlink,capitalize]{cleveref}

\usepackage{listings}
\lstset{
%  numbers=left,
%  firstnumber=1,
}

\usepackage{hyperref}

\usepackage[vlined,linesnumbered]{algorithm2e}
  \SetKwProg{Fn}{Function}{}{}

\usepackage{verbatim}

\usepackage{geometry}
  \geometry{hmargin=3.5cm,vmargin=3.5cm}

\usepackage{amsfonts}
\usepackage{amsthm}
\usepackage{minted}
\usepackage{amsmath}

\title{Calcul parallèle - TP 2: Correcteur de mots}
\author{Adrien Stalain et Vincent Tourneur}

\begin{document}

\maketitle

\section{Structure de données principale}

Les fichiers qui sont décrits dans cette section sont:
\begin{itemize}
	\item \texttt{src/atd/advanced\_trie.hpp}
	\item \texttt{src/atd/advanced\_trie.cpp}
\end{itemize}

Pour stoquer les mots du dictionnaire, nous utilisons un \texttt{trie}. Tous
les algorithmes utilisés pour la recherche de mots, l'ajout et la suppression
n'utilisent aucun vérou (\texttt{lock-free}). Nous utilisons à la place des
\texttt{std::atomic}, en spécifiant le \texttt{memory order} à chaque
écriture/lecture pour avoir de bonnes performances.

Chaque noeud contient une lettre, un booléen qui indique si le mot formé depuis
la racine de l'arbre est valide et la liste de ses fils:
\begin{minted}{c++}
class advanced_trie
{
	using childs_t = std::array<std::atomic<advanced_trie*>, 'z' - 'a' + 1>;
	childs_t childs_;
	advanced_trie* parent_;
	char my_char_; // Only if there is a parent
	std::atomic<bool> word_end_;
}
\end{minted}

La liste des fils est enregistrée dans un \texttt{std::array}, qui a comme
taille 26, c'est-à-dire le nombre maximum de fils possible, si la node n'as pas 
26 fils, le tableau est \texttt{null-terminated}.

Lors d'un ajout de mot, nous parcourons l'arbre, et à partir du moment où le
noeud recherché n'existe pas, nous créons entièrement la nouvelle branche avant
de l'attacher au reste de l'arbre. Cette opération est faite de façon atomique,
avec un \texttt{compare\_exchange\_strong}, au cas où un autre \texttt{thread}
a rattaché une autre branche au même endroit juste avant. Le booléen
\texttt{word\_end\_} est mis à \texttt{true} sur la feuille de cette branche.
Si la branche existait déjà entièrement, le booléen du dernier noeud de la branche
est mis à \texttt{true}, il n'est pas nécessaire de créer des noeuds.

Pour l'opération de suppression, nous parcourons la branche de l'arbre correspondant
au mot, puis le booléen \texttt{word\_end\_} est mis à \texttt{false}. Si la
branche n'existe pas, la fonction s'arrête.

L'opération de recherche est décrite dans la section suivante.

\section{Algorithme principal}

Les fichiers qui sont décrits dans cette section sont:
\begin{itemize}
	\item \texttt{src/advanced\_trie\_dictionary.cpp}
	\item \texttt{src/atd/task.hpp}
	\item \texttt{src/atd/task\_stack.hpp}
\end{itemize}

La recherche de mots est faite en calculant la distance de Levenstein entre le
mot recherché et les mots contenus dans l'arbre. Les calculs sont faits en
remplissant des vecteurs, dont la case $n$ correspond à la distance avec le
préfixe de $n$ lettres du mot recherché. Nous commençons par la racine de
l'arbre, qui correspond au mot vide.

Descendre dans l'arbre correspond à remplir un nouveau vecteur. Pour cela,
nous avons besoin des deux précédents vecteurs, car pour remplir la prochaine
case du vecteur courant, nous avons besoin de la valeur de la case précédente,
des valeurs des cases actuelle et précédente du précédent vecteur, et au cas
où un échange de lettre est requis, de la deuxième précédente case du deuxième
précedent vecteur.

Nous appelons une tâche l'action de remplir un vecteur, et ces tâches sont
stoquées dans des piles de tâches pour permettre de faire facilement des
parcours autres que profondeur de l'arbre. Chaque tâche, après avoir remplit
son vecteur, va créer des nouvelles tâches qui correspondent aux fils du noeud.

Nous stoquons les tâches dans deux piles différentes: lorsqu'une tâche calcule
la dernière valeur de son vecteur (qui correspond à la distance entre le mot
recherché et le mot qui correspond au noeud de l'arbre), celle-ci peut avoir
seulement deux valeurs: soit la même valeur que la dernière case du vecteur
précedent, soit cette valeur plus un.

Dans le premier cas, la tâche est ajoutées dans la première pile, et dans le
second cas, dans la deuxième pile. L'ordre de traitement des tâches est le
suivant: les tâches de la première pile sont exécutées, jusqu'à ce qu'elle soit
vide, puis les deux piles sont échangées (donc la première pile reçoit le contenu
de la deuxième pile, et cette dernière devient la pile vide).

Lorsque la tâche courrante correspond à un mot valide du dictionnaire,
nous testons si sa distance est plus faible que la meilleure connue, et
si c'est le cas nous l'enregistrons. S'il ne reste aucune tâche à explorer
qui pourrait faire baisser la distance minimale connue, nous arrêtons le parcours.
La distance la plus faible et le mot associé sont retournés.

Cet ordre de traitement nous permet d'éviter d'aller parcourir des longues
branches de l'arbre qui contiennent des mots très différents de celui recherché.
Cela permet des recherches rapides, surtout quand la distance est faible.

\section{Optimisations mémoire}
Les fichiers qui sont décrits dans cette section sont:
\begin{itemize}
	\item \texttt{src/atd/line.hpp}
	\item \texttt{src/atd/linepool.hpp}
	\item \texttt{src/atd/linepool.cpp}
\end{itemize}

Notre parcours de \texttt{trie} est relativement coûteux en mémoire
(jusqu'à $26^{dist}$ vecteurs en mémoire), et il y a de nombreuses allocations
mémoires pour peu de déallocations. C'est pour cela que l'on a choisi de faire
un sytème d'allocation personnalisé.

L'allocateur personalisé est basé sur une \texttt{memory pool},
il nous donne plusieurs avantages:
\begin{itemize}
  \item On réduit le coup d'allocation de la biblothèque standard.
  \item On n'utilise pas de \texttt{std::shared\_ptr} et leur coût
	  (\texttt{thread safety} / \texttt{reference counting}).
  \item On n'appelle pas les destructeurs.
\end{itemize}

\end{document}
